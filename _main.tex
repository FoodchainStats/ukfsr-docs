% Options for packages loaded elsewhere
\PassOptionsToPackage{unicode}{hyperref}
\PassOptionsToPackage{hyphens}{url}
%
\documentclass[
]{book}
\usepackage{amsmath,amssymb}
\usepackage{iftex}
\ifPDFTeX
  \usepackage[T1]{fontenc}
  \usepackage[utf8]{inputenc}
  \usepackage{textcomp} % provide euro and other symbols
\else % if luatex or xetex
  \usepackage{unicode-math} % this also loads fontspec
  \defaultfontfeatures{Scale=MatchLowercase}
  \defaultfontfeatures[\rmfamily]{Ligatures=TeX,Scale=1}
\fi
\usepackage{lmodern}
\ifPDFTeX\else
  % xetex/luatex font selection
\fi
% Use upquote if available, for straight quotes in verbatim environments
\IfFileExists{upquote.sty}{\usepackage{upquote}}{}
\IfFileExists{microtype.sty}{% use microtype if available
  \usepackage[]{microtype}
  \UseMicrotypeSet[protrusion]{basicmath} % disable protrusion for tt fonts
}{}
\makeatletter
\@ifundefined{KOMAClassName}{% if non-KOMA class
  \IfFileExists{parskip.sty}{%
    \usepackage{parskip}
  }{% else
    \setlength{\parindent}{0pt}
    \setlength{\parskip}{6pt plus 2pt minus 1pt}}
}{% if KOMA class
  \KOMAoptions{parskip=half}}
\makeatother
\usepackage{xcolor}
\usepackage{longtable,booktabs,array}
\usepackage{calc} % for calculating minipage widths
% Correct order of tables after \paragraph or \subparagraph
\usepackage{etoolbox}
\makeatletter
\patchcmd\longtable{\par}{\if@noskipsec\mbox{}\fi\par}{}{}
\makeatother
% Allow footnotes in longtable head/foot
\IfFileExists{footnotehyper.sty}{\usepackage{footnotehyper}}{\usepackage{footnote}}
\makesavenoteenv{longtable}
\usepackage{graphicx}
\makeatletter
\def\maxwidth{\ifdim\Gin@nat@width>\linewidth\linewidth\else\Gin@nat@width\fi}
\def\maxheight{\ifdim\Gin@nat@height>\textheight\textheight\else\Gin@nat@height\fi}
\makeatother
% Scale images if necessary, so that they will not overflow the page
% margins by default, and it is still possible to overwrite the defaults
% using explicit options in \includegraphics[width, height, ...]{}
\setkeys{Gin}{width=\maxwidth,height=\maxheight,keepaspectratio}
% Set default figure placement to htbp
\makeatletter
\def\fps@figure{htbp}
\makeatother
\setlength{\emergencystretch}{3em} % prevent overfull lines
\providecommand{\tightlist}{%
  \setlength{\itemsep}{0pt}\setlength{\parskip}{0pt}}
\setcounter{secnumdepth}{5}
\usepackage{booktabs}
\ifLuaTeX
  \usepackage{selnolig}  % disable illegal ligatures
\fi
\usepackage[]{natbib}
\bibliographystyle{plainnat}
\IfFileExists{bookmark.sty}{\usepackage{bookmark}}{\usepackage{hyperref}}
\IfFileExists{xurl.sty}{\usepackage{xurl}}{} % add URL line breaks if available
\urlstyle{same}
\hypersetup{
  pdftitle={UK Food Security Report Guidance},
  pdfauthor={Food Statistics Team},
  hidelinks,
  pdfcreator={LaTeX via pandoc}}

\title{UK Food Security Report Guidance}
\author{Food Statistics Team}
\date{Last updated: 2023-12-21}

\begin{document}
\maketitle

{
\setcounter{tocdepth}{1}
\tableofcontents
}
\hypertarget{about}{%
\chapter{About}\label{about}}

This is an experimental book/website for documentation and guidance relating to the UKFSR. The source is \href{https://github.com/FoodchainStats/ukfsr-docs}{here}.

\hypertarget{usage}{%
\section{Usage}\label{usage}}

The plan is to capture everything related to UKFSR2024 production based on lessons from UKFSR2021 which might be useful, and end up with a repository to pick up again once UKFSR2027 is being planned.

We aim for a logical structure, but there may be some nuggets dropped in unusual places. Its a brain dump!

We are working in the open, so this is not a Defra Sharepoint repository but a public Github one. We know that this means its not quite so accessible for policy professionals, but we will work on this. We are allowed to have internal conversations. We don't put anything in here that we are not prepared to make public - its for future teams, to make sure we've captured all the useful info to make their lives easier. If there are sensitive messages/lessons, we store them somewhere else.

This approach mirrors our approach to data and the production process: we are storing the relevant code we use to wrangle data and produce graphics in a similar repository. Data that is not open will of course not be shared.

Being as open and reproducible as we can is one of the current ambitions for official statistics production. This site is part of our way of trying to meet that ambition. That does not mean we will not make necessary internal decisions as needed - those will be recorded appropriately and documented safely elsewhere.

\hypertarget{text}{%
\chapter{Text style guide}\label{text}}

For chart design guidance, see Section \ref{chart}.

\hypertarget{gov.uk-style-guide}{%
\section{GOV.UK style guide}\label{gov.uk-style-guide}}

Guidance for all content published on gov.uk is available \href{https://www.gov.uk/guidance/style-guide}{here}

\hypertarget{formatting}{%
\section{Formatting}\label{formatting}}

\begin{itemize}
\tightlist
\item
  use appropriate styles, particularly for headings. These needed when building the master publications
\end{itemize}

\hypertarget{ukfsr-specific-guidance}{%
\section{UKFSR specific guidance}\label{ukfsr-specific-guidance}}

Some of this may replicate the GDS advice.

\hypertarget{spellingword-choice}{%
\subsection{Spelling/Word choice}\label{spellingword-choice}}

\begin{itemize}
\item
  Say ``UKFSR'' or ``this report'', not ``the Report''.
\item
  Use `coronavirus (COVID-19)' in the text at first mention, then `COVID-19' after that.
\item
  government, UK government (not capitalised unless it's Welsh or Scottish Government)
\item
  other government departments
\item
  Use `the' when talking about `the FSA', but don't use `the' when talking about `FSS'
\item
  FSA and FSS can be referred to collectively as ``UK food safety bodies'', but not ``food standards agencies.''
\item
  Capitals when talking about ethnicities, e.g.~`White', `Black'
\item
  Disabled people and not people with disabilities
\item
  For the Ukraine War, refer to it as the ``Ukraine War'', ``the war in Ukraine'' or ``Russia-Ukraine War''.
\item
  Talking about Brexit:

  \begin{itemize}
  \item
    You can use the term `Brexit' to provide historical context, but it's better to use specific dates where possible. For example, use:

    \begin{itemize}
    \tightlist
    \item
      `31 December 2020' rather than `Brexit' or `when the UK left the EU'
    \item
      `before 31 December 2020' rather than `during the transition period'
    \item
      `after 1 January 2021' rather than `after the transition period'
    \end{itemize}
  \end{itemize}
\item
  Dates: do not use a comma between the month and year: 4 June 2017
\item
  white paper (lower case)
\item
  Be very careful with the word ``affordable''. Food is more affordable if it's cheaper relative to incomes and other factors -- if the price has gone down (but other factors mean it's harder to afford) it's simply \emph{cheaper}. The general argument of the UKFSR is that over the last decade food \emph{has got cheaper, but not more affordable} -- so check this!
\item
  Words to avoid:

  \begin{itemize}
  \tightlist
  \item
    Robust
  \item
    Overarching
  \item
    Strengthen (unless we are actually strengthening an architectural structure)
  \item
    Tackling\\
  \item
    Going forward
  \item
    In order to (superfluous, never use it)
  \item
    impact (do not use this as a synonym for have an effect on, or influence)
  \item
    facilitate (instead, say something specific about how you're helping)
  \item
    focusing
  \item
    key (unless it unlocks something. A subject/thing is not key - it's probably important)
  \end{itemize}
\end{itemize}

\hypertarget{references}{%
\subsection{References}\label{references}}

Include a reference in-text after the relevant sentence/paragraph. References should follow the style guide. When writing a reference:

\begin{itemize}
\tightlist
\item
  do not use italics
\item
  use single quote marks around titles
\item
  write out abbreviations in full: page not p, Nutrition Journal not Nutr J.
\item
  use plain English, for example use `and others' not `et al'
\item
  do not use full stops after initials or at the end of the reference
\end{itemize}

\emph{References can also be included as footnotes, particularly where too long or unwieldy for convenience in text.}

If the reference is available online, make the title a link and include the date you accessed the online version. For example:

\begin{itemize}
\item
  Although food availability is increasing in low and middle-income countries, fruit and vegetables are still high-value items, meaning fats and sweeteners will make up large parts of the increase in consumption (FAO. \href{https://www.fao.org/3/cb5332en/cb5332en.pdf}{`OECD-FAO Agricultural Outlook 2021-2030'} 2021).
\item
  There is a recurring reference through multiple themes to AUK, check for consistency to make sure it is referenced as follows: (Defra. \href{https://assets.publishing.service.gov.uk/government/uploads/system/uploads/attachment_data/file/1056618/AUK2020_22feb22.pdf}{`Agriculture in the United Kingdom 2020'} 2020)
\item
  Example given on gov.uk: Corallo AN and others. \href{https://www.sciencedirect.com/journal/health-policy}{`A systematic review of medical practice variation in OECD countries'} Health Policy 2014: volume 114, pages 5 to 14.
\end{itemize}

\hypertarget{general}{%
\subsection{General}\label{general}}

\hypertarget{acronyms}{%
\subsubsection{Acronyms}\label{acronyms}}

\begin{itemize}
\tightlist
\item
  Write out acronym for the first time in each theme, put abbreviation in brackets, then use abbreviation going forwards. This means acronyms should be re-introduced anew between the introduction and themes.
\end{itemize}

\hypertarget{labelling-data}{%
\subsubsection{Labelling data}\label{labelling-data}}

\begin{itemize}
\tightlist
\item
  Each data set (table, bar chart, etc.) should be labelled as `Figure' with the corresponding indicator number and a letter. For instance, for data in theme 1, this could look like `Figure 1.1.2a', Figure `1.1.2b', etc.
\end{itemize}

\hypertarget{headings}{%
\subsubsection{Headings}\label{headings}}

Each indicator should have the following headings (check for consistent spelling):

\begin{itemize}
\item
  \textbf{Headline}

  \begin{itemize}
  \tightlist
  \item
    Brief summary of the key findings in the indicator -- seen from a ``what this means for food security lens'' (ie, `the UK produces around 80\% of the wheat it consumes; average production is stable with some fluctuations due to weather' rather than `the UK produces x million tons of wheat')
  \end{itemize}
\item
  \textbf{Context and Rationale}

  \begin{itemize}
  \tightlist
  \item
    Background information for the data and explanation why this data matters for UK food security
  \end{itemize}
\item
  \textbf{Data and Assessment}

  \begin{itemize}
  \tightlist
  \item
    Data and source

    \begin{itemize}
    \tightlist
    \item
      If multiple sources, list as follows:\\
    \item
      Source: FSA; FSS
    \end{itemize}
  \item
    Description of what the data in the `Data and Assessment' section is showing
  \end{itemize}
\item
  \textbf{Rating and Trends / Direction of Travel}

  \begin{itemize}
  \tightlist
  \item
    State any visible trends in the data: ie, things appear to be stable, or to fluctuate, or have a clear positive upward/downward trend, or it's not clear from available data. Note also any important external factors.
  \end{itemize}
\end{itemize}

\hypertarget{the-voice}{%
\subsubsection{The `Voice'}\label{the-voice}}

\begin{itemize}
\tightlist
\item
  Avoid using sentences such as `We produce x amount of wheat'. The tone should be more neutral, i.e.~write `The UK produces x amount of wheat'.
\end{itemize}

\hypertarget{e.g.-i.e.-etc.}{%
\subsubsection{e.g.~/ i.e.~/ etc.}\label{e.g.-i.e.-etc.}}

\begin{itemize}
\tightlist
\item
  \textbf{e.g.} can sometimes be read aloud as `egg' by screen reading software. Instead use `for example' or `such as' or `like' or `including' - whichever works best in the specific context.
\item
  \textbf{etc} can usually be avoided. Try using `for example' or `such as' or `like' or `including'. Never use etc at the end of a list starting with these words.
\item
  \textbf{ie} - used to clarify a sentence - is not always well understood. Try (re)writing sentences to avoid the need to use it. If that is not possible, use an alternative such as `meaning' or `that is'.
\end{itemize}

\hypertarget{symbols}{%
\subsubsection{Symbols}\label{symbols}}

\begin{itemize}
\tightlist
\item
  `\&' use `and', e.g.~Context and rationale
\item
  `/' use `and', e.g 2007/2008 = 2007 and 2008
\item
  `-' use `to' (for example in dates), e.g.~2011-14 = 2011 to 2014
\end{itemize}

\hypertarget{numbers}{%
\subsubsection{Numbers}\label{numbers}}

\begin{itemize}
\tightlist
\item
  Write all other numbers in numerals (including 2 to 9) except where it's part of a common expression like `one or two of them' where numerals would look strange.
\item
  Use a \% sign for percentages: 50\%
\item
  Use `500 to 900' and not `500-900' (except in tables)
\end{itemize}

\hypertarget{quotation-marks}{%
\subsubsection{Quotation marks:}\label{quotation-marks}}

Use single quotes

\begin{itemize}
\tightlist
\item
  in headlines
\item
  for unusual terms
\item
  when referring to words
\item
  when referring to publications
\end{itemize}

Use double quotes:

\begin{itemize}
\tightlist
\item
  Use double quotes in body text for direct quotations
\end{itemize}

\hypertarget{sentence-length}{%
\subsubsection{Sentence length}\label{sentence-length}}

\begin{itemize}
\tightlist
\item
  Do not use long sentences. Check sentences with more than 25 words to see if you can split them to make them clearer.
\end{itemize}

\hypertarget{chart}{%
\chapter{Chart Style Guide}\label{chart}}

This section will cover guidance on chart design and themes. For the text
styleguide, see Chapter \ref{text}. There is general guidance on chart design
produced by the \href{https://analysisfunction.civilservice.gov.uk/policy-store/data-visualisation-charts/}{Analysis Function}.

\hypertarget{colours}{%
\section{Colours}\label{colours}}

The UKFSR uses the Analysis Function palette as recommended \href{https://analysisfunction.civilservice.gov.uk/policy-store/data-visualisation-colours-in-charts/}{here}. There is an R package, \texttt{afcolours} available on CRAN, which helps with implementing the \href{https://best-practice-and-impact.github.io/afcolours/}{palette}.

\hypertarget{data-guide}{%
\chapter{Data guide}\label{data-guide}}

How we prepare and organise data.

\hypertarget{theme-guides}{%
\chapter{Theme guides}\label{theme-guides}}

\hypertarget{theme-1}{%
\section{Theme 1}\label{theme-1}}

\hypertarget{theme-2}{%
\section{Theme 2}\label{theme-2}}

\hypertarget{theme-3}{%
\section{Theme 3}\label{theme-3}}

\hypertarget{theme-4}{%
\section{Theme 4}\label{theme-4}}

\hypertarget{theme-5}{%
\section{Theme 5}\label{theme-5}}

\hypertarget{web-publishing-guidance}{%
\chapter{Web publishing guidance}\label{web-publishing-guidance}}

\hypertarget{logistics}{%
\section{Logistics}\label{logistics}}

Web publishing for statistics is handled by a small number of people who are accredited to use whitehall Publisher. We are fortunate to have two publishers in the food team. But there is an overhead. \textbf{Factor in prep for web time! (ideally 1 week)}

\begin{itemize}
\tightlist
\item
  web team to commandeer additional resource to cover secondeyes etc process during web print conversion
\end{itemize}

\hypertarget{converting-word-documents-to-markdown}{%
\section{Converting Word documents to markdown}\label{converting-word-documents-to-markdown}}

R code for

\begin{verbatim}
library(rmarkdown)
pandoc_convert(input = 'FILENAME.docx',
               to="markdown_mmd",
               output = "FILENAME.md", 
               options = c("--wrap=none",
                           "--reference-links",
                           "--extract-media=./images/"))
\end{verbatim}

\hypertarget{graphics}{%
\section{Graphics}\label{graphics}}

Images must be 960 pixels wide by 640 pixels high at 72 dpi.
SVGs can be any size and do not need resizing before uploading.

\hypertarget{other-useful-things}{%
\section{Other useful things}\label{other-useful-things}}

\hypertarget{acronym-markdown}{%
\subsection{Acronym markdown}\label{acronym-markdown}}

Add to end of file, each one on a newline

\begin{verbatim}
*[UKFSR]: UK Food Security Report
\end{verbatim}

\hypertarget{physical-print-edition}{%
\chapter{Physical print edition}\label{physical-print-edition}}

The logistics of print will depend on the service used to create the print edition. For UKFSR2021 the printers needed a PDF of the final document to turn into the final laid report.

\textbf{Factor in print publication time! (ideally 1 week)}

\hypertarget{final-word-version}{%
\section{Final Word version}\label{final-word-version}}

The assumption is that the final version will be contained in multiple Word files: one for each theme plus introductions, appendices, glossary, etc. Once these are finalised a master print version can be produced.

\begin{itemize}
\tightlist
\item
  `Lock' the content: set a deadline and get everyone out of the Sharepoint documents
\item
  copy the final section docs into a new folder: this folder is the master print content
\item
  get a copy of the Defra doc template which has a default front and back cover
\item
  check and secure an ISBN/Defra document number
\item
  check any standard copyright notice text
\item
  check other cover/back cover/inside cover text: last time there was specific wording like `laid before the HoC Library on \emph{date}' that needed to conform with protocol.
\end{itemize}

With this cover sheet insert the section docs as `subdocuments', in outline view. This allows the building of the master table of contents which covers the whole report. Insert the TOC and set up the Contents pages. ONce all the content is in place, we can turn ro formatting for print:

\begin{itemize}
\tightlist
\item
  Margins: each section of the report needs to be formatted with mirror margins so that there is a larger gutter where the binding will be. Otherwise the content will get lost in the printed doc.
\item
  Page numbering: inside and outside so that they appear on the right and left hand side of the printed page. You need to set page numbers to inherit from each previous section so that they flow from 1-n and don't reset at the start of each child doc.
\item
  Pagination: this was a bit of a fiddle last time. Make sure that the content is settled (You probably want to eyeball the content for any errors with heading levels etc and fix all content snags before doing this or else you'll have to do it again). Refresh the TOC, and then go through inserting white space where necessary to ensure page breaks don't occur in unfortunate places.
\end{itemize}

At this point the document should be ready for PDF conversion.

\hypertarget{create-pdf}{%
\section{Create PDF}\label{create-pdf}}

At this point it should be easy: export the master Word doc to PDF (PDF/A I think: check!). Worth checking accessibility even though the HTML version will be the truly accessible version. One thing worth checking: metadata. Check the author, organisation etc metadata in the word doc is what you want in the PDF version.

\hypertarget{accessibility}{%
\section{Accessibility}\label{accessibility}}

There was an unfulfilled request for a large print version last time. With a bit of luck, any MPs with such requirements will accept the HTML version as accessible, zoomable and screen reader friendly. If not, repeat the document conversion process and edit the style definitions to up the font size. I think there might be practical printing issues which mean it would have to be printed as multiple documents. UKFSR at 300 odd pages I think was approaching the limit of what perfect binding print can handle. Speak to the print supplier about these issues.

\hypertarget{appendix-appendix}{%
\appendix}


\hypertarget{miscellaneous}{%
\chapter{Miscellaneous}\label{miscellaneous}}

\hypertarget{legal-basis}{%
\section{Legal basis}\label{legal-basis}}

\href{https://www.legislation.gov.uk/ukpga/2020/21/part/2/chapter/1/enacted}{Agriculture Act 2020}

\hypertarget{osr-feedback-on-ukfsr2021}{%
\section{OSR feedback on UKFSR2021}\label{osr-feedback-on-ukfsr2021}}

\href{https://osr.statisticsauthority.gov.uk/correspondence/mark-pont-to-ian-lonsdale-united-kingdom-food-security-report/}{Here.}

\hypertarget{publication-deadline}{%
\section{Publication deadline}\label{publication-deadline}}

Insert legal guidance

\hypertarget{general-legal-guidance-from-2024}{%
\section{General legal guidance from 2024}\label{general-legal-guidance-from-2024}}

\hypertarget{late-laying-of-the-report}{%
\subsection{Late laying of the report}\label{late-laying-of-the-report}}

The starting point is that there is a clear statutory duty to lay the report by the last sitting day before Christmas day (s19(3) of the Agriculture Act 2020). Although the Act states ``In this section ``relevant day'' means the last day before 25 December 2021 which is a sitting day for both Houses of Parliament.'', I think we can safely assume that the last sitting day before 25 December 2024 is correct. If the report is laid later than this date there would be a breach of this statutory duty. The risk of a legal challenge being brought (our assessment is that there would be a low risk (less than 30\%)) would depend on the circumstances that caused the delay and the length of the delay. However, if it is brought the chances of successful challenge are medium-high (50-70\%) given that there is a clear statutory duty to publish by a certain date. We could perhaps mitigate this depending on the circumstances which would lower the risk. For example, the Courts would likely look more favourably on SoS if the reason for delay was an election. However, if you know there will be an election then steps should be taken to avoid delay as much as possible. In short, unavoidable circumstances would look more favourable to the Courts than avoidable ones.

\hypertarget{environmental-principles-policy}{%
\subsection{Environmental principles policy}\label{environmental-principles-policy}}

We have considered whether the Environmental principles policy would apply here. Section 17(5) of the Environment Act 2021 sets out 5 principles that ministers will need to consider when making policy, namely:

\begin{itemize}
\tightlist
\item
  integration principle
\item
  prevention principle
\item
  rectification at source principle
\item
  polluter pays principle
\item
  precautionary principle
\end{itemize}

Whilst any particular concerns arising from considerations in the EPP statement will likely not apply when collecting data and compiling the report, any policy development that follows on from the reports' findings should have due regard to the principles. This is just something to flag at this stage.

\hypertarget{general-considerations}{%
\subsection{General considerations}\label{general-considerations}}

Lastly, and as a general legal point, if the contents of the report were to be challenged, the ground would likely be rationality. It is therefore important that information sources are referenced correctly and checked. The 2021 report will of course provide an excellent starting point and seeing as that was not challenged, adopting a similar approach would be sensible.

  \bibliography{book.bib,packages.bib}

\end{document}
