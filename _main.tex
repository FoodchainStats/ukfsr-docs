% Options for packages loaded elsewhere
\PassOptionsToPackage{unicode}{hyperref}
\PassOptionsToPackage{hyphens}{url}
%
\documentclass[
]{book}
\usepackage{amsmath,amssymb}
\usepackage{iftex}
\ifPDFTeX
  \usepackage[T1]{fontenc}
  \usepackage[utf8]{inputenc}
  \usepackage{textcomp} % provide euro and other symbols
\else % if luatex or xetex
  \usepackage{unicode-math} % this also loads fontspec
  \defaultfontfeatures{Scale=MatchLowercase}
  \defaultfontfeatures[\rmfamily]{Ligatures=TeX,Scale=1}
\fi
\usepackage{lmodern}
\ifPDFTeX\else
  % xetex/luatex font selection
\fi
% Use upquote if available, for straight quotes in verbatim environments
\IfFileExists{upquote.sty}{\usepackage{upquote}}{}
\IfFileExists{microtype.sty}{% use microtype if available
  \usepackage[]{microtype}
  \UseMicrotypeSet[protrusion]{basicmath} % disable protrusion for tt fonts
}{}
\makeatletter
\@ifundefined{KOMAClassName}{% if non-KOMA class
  \IfFileExists{parskip.sty}{%
    \usepackage{parskip}
  }{% else
    \setlength{\parindent}{0pt}
    \setlength{\parskip}{6pt plus 2pt minus 1pt}}
}{% if KOMA class
  \KOMAoptions{parskip=half}}
\makeatother
\usepackage{xcolor}
\usepackage{longtable,booktabs,array}
\usepackage{calc} % for calculating minipage widths
% Correct order of tables after \paragraph or \subparagraph
\usepackage{etoolbox}
\makeatletter
\patchcmd\longtable{\par}{\if@noskipsec\mbox{}\fi\par}{}{}
\makeatother
% Allow footnotes in longtable head/foot
\IfFileExists{footnotehyper.sty}{\usepackage{footnotehyper}}{\usepackage{footnote}}
\makesavenoteenv{longtable}
\usepackage{graphicx}
\makeatletter
\def\maxwidth{\ifdim\Gin@nat@width>\linewidth\linewidth\else\Gin@nat@width\fi}
\def\maxheight{\ifdim\Gin@nat@height>\textheight\textheight\else\Gin@nat@height\fi}
\makeatother
% Scale images if necessary, so that they will not overflow the page
% margins by default, and it is still possible to overwrite the defaults
% using explicit options in \includegraphics[width, height, ...]{}
\setkeys{Gin}{width=\maxwidth,height=\maxheight,keepaspectratio}
% Set default figure placement to htbp
\makeatletter
\def\fps@figure{htbp}
\makeatother
\setlength{\emergencystretch}{3em} % prevent overfull lines
\providecommand{\tightlist}{%
  \setlength{\itemsep}{0pt}\setlength{\parskip}{0pt}}
\setcounter{secnumdepth}{5}
\usepackage{booktabs}
\ifLuaTeX
  \usepackage{selnolig}  % disable illegal ligatures
\fi
\usepackage[]{natbib}
\bibliographystyle{plainnat}
\IfFileExists{bookmark.sty}{\usepackage{bookmark}}{\usepackage{hyperref}}
\IfFileExists{xurl.sty}{\usepackage{xurl}}{} % add URL line breaks if available
\urlstyle{same}
\hypersetup{
  pdftitle={UK Food Security Report Guidance},
  pdfauthor={Food Statistics Team},
  hidelinks,
  pdfcreator={LaTeX via pandoc}}

\title{UK Food Security Report Guidance}
\author{Food Statistics Team}
\date{Last updated: 2023-12-04}

\begin{document}
\maketitle

{
\setcounter{tocdepth}{1}
\tableofcontents
}
\hypertarget{about}{%
\chapter{About}\label{about}}

This is an experimental book/website for documentation and guidance relating to the UKFSR. The source is \href{https://github.com/FoodchainStats/ukfsr-docs}{here}.

\hypertarget{usage}{%
\section{Usage}\label{usage}}

\hypertarget{text}{%
\chapter{Text style guide}\label{text}}

Copy elements of the UKFSR2021 style guide here. For chart design guidance, see
Chapter \ref{chart}.

\hypertarget{gov.uk-style-guide}{%
\section{GOV.UK style guide}\label{gov.uk-style-guide}}

Guidance for all content published on gov.uk is available \href{https://www.gov.uk/guidance/style-guide}{here}

\hypertarget{chart}{%
\chapter{Chart Style Guide}\label{chart}}

This section will cover guiidance on chart design and themes. For the txt
styleguide, see Chapter \ref{text}.

\hypertarget{data-guide}{%
\chapter{Data guide}\label{data-guide}}

How we prepare and organise data.

\hypertarget{theme-guides}{%
\chapter{Theme guides}\label{theme-guides}}

\hypertarget{theme-1}{%
\section{Theme 1}\label{theme-1}}

\hypertarget{theme-2}{%
\section{Theme 2}\label{theme-2}}

\hypertarget{theme-3}{%
\section{Theme 3}\label{theme-3}}

\hypertarget{theme-4}{%
\section{Theme 4}\label{theme-4}}

\hypertarget{theme-5}{%
\section{Theme 5}\label{theme-5}}

  \bibliography{book.bib,packages.bib}

\end{document}
